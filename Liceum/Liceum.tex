\documentclass{article}
\usepackage[T1]{fontenc}
\usepackage[utf8]{inputenc}
\usepackage{cancel}
\usepackage{amsmath}
\usepackage{amssymb}
\usepackage{polski}
\hbadness=10000 % Suppress underfull hbox warnings

\title{Liceum}
\author{Wiktoria Borek}
\date{}

\begin{document}

\maketitle

\section{Logarytmy}
\subsection{Zadanie}
Wykaż, że dla $x>1$ zachodzi równość: \\
$2\log_2{(x^2-1)} - 3\log_2{(x-1)} = \log_2{\frac{x^2+2x+1}{x-1}}$
\subsubsection*{Rozwiązanie}
$
  D: x>1 \\
    \\
    L = 2\log_2{(x^2-1)} - 3\log_2{(x-1)} \\
    P = \log_2{\frac{x^2+2x+1}{x-1}} \\
    \\
    L = \log_2{(x^2-1)^2} - \log_2{(x-1)^3} = \log_2{\frac{(x^2-1)^2}{(x-1)^3}} = \log_2{\frac{\cancel{(x-1)^2}(x+1)^2}{\cancel{(x-1)^2}(x-1)}} = \\
    = \log_2{\frac{x^2+2x+1}{x-1}} = P \quad \blacksquare
$
\subsection{Zadanie}
Wykaż, że $\log_{17}{19} : \log_{18}{19} = \log_{16}{18} \cdot \log_{17}{16}$.
\subsubsection*{Rozwiązanie}
$
  \log_{17}{19} \cdot \log_{19}{18} = \log_{17}{16} \cdot \log_{16}{18} \\
  \log_{17}{\cancel{19}} \cdot \log_{\cancel{19}}{18} = \log_{17}{\cancel{16}} \cdot \log_{\cancel{16}}{18} \\
  \log_{17}{18} = \log_{17}{18} \\
  L = P \quad \blacksquare
$
\subsection{Zadanie}
Dane są liczby $a=\log_3{2}$ oraz $b=\log_{2^{2024}}{3^{1012}} + \log_{\frac{1}{3}}{54}$. Wyraź liczbę $b$ za pomocą liczby $a$.
\subsubsection*{Rozwiązanie}
$
  b = \frac{1}{2} \cdot \log_2{3} - \log_3{(27 \cdot 2)} = \frac{1}{2} \cdot \frac{1}{a} - (\log_3{3^3} + a) = \frac{1}{2a} - 3 - a
$
\end{document}