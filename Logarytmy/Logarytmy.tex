\documentclass{article}
\usepackage[fleqn]{amsmath}
\usepackage{parskip}
\usepackage{titlesec}
\usepackage{cancel}
\usepackage{amssymb}
\usepackage{polski}

\titleformat{\section}[block]{\normalfont\large\bfseries}{\thesection}{0pt}{}
\titlespacing*{\section}{0pt}{3ex}{2ex}

\makeatletter
\renewcommand{\@maketitle}{
  \newpage
  \null
  \vskip 2em%
  \begin{flushleft}%
    {\LARGE\bfseries \@title \par}%
    \vskip 1.5em%
    {\large \@date}%
  \end{flushleft}%
  \par
  \vskip 1.5em}
\makeatother

\title{Logarytmy}
\date{}

\begin{document}

\maketitle

\section*{Zadanie 1}
Wykaż, że dla $x>1$ zachodzi równość: \\
$2\log_2{(x^2-1)} - 3\log_2{(x-1)} = \log_2{\frac{x^2+2x+1}{x-1}}$
\begin{gather*}
	D: x>1 \\
    \\
    L = 2\log_2{(x^2-1)} - 3\log_2{(x-1)} \\
    P = \log_2{\frac{x^2+2x+1}{x-1}} \\
    \\
    L = \log_2{(x^2-1)^2} - \log_2{(x-1)^3} = \log_2{\frac{(x^2-1)^2}{(x-1)^3}} = \log_2{\frac{\cancel{(x-1)^2}(x+1)^2}{\cancel{(x-1)^2}(x-1)}} = \\
    = \log_2{\frac{x^2+2x+1}{x-1}} = P \quad \blacksquare
\end{gather*}

\section*{Zadanie 2}
Wykaż, że $\log_{17}{19} : \log_{18}{19} = \log_{16}{18} \cdot \log_{17}{16}$.
\begin{gather*}
  \log_{17}{19} \cdot \log_{19}{18} = \log_{17}{16} \cdot \log_{16}{18} \\
  \log_{17}{\cancel{19}} \cdot \log_{\cancel{19}}{18} = \log_{17}{\cancel{16}} \cdot \log_{\cancel{16}}{18} \\
  \log_{17}{18} = \log_{17}{18} \\
  L = P \quad \blacksquare
\end{gather*}

\end{document}